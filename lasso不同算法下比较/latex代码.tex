\documentclass[twoside]{article}
\usepackage{multirow}  
\usepackage{ctex}
\usepackage{amssymb, amsmath}
\usepackage{mathrsfs}
\usepackage{booktabs}
\usepackage{graphicx}
\usepackage{enumitem}  % 列表优化
\usepackage{titlesec}  % 标题格式自定义
\usepackage{tikz} % 绘图支持
\usepackage{multicol} % 新增:解决multicols未定义报错
\usetikzlibrary{shapes,arrows,positioning} % 绘图库

% 定理/定义等环境定义(标准模板)
\newtheorem{theorem}{Theorem}
\newtheorem{corollary}{Corollary}
\newtheorem{lemma}{Lemma}
\newtheorem{example}{Example}
\newtheorem{remark}{Remark}
\newtheorem{proposition}{Proposition}
\newtheorem{problem}{Problem}
\newtheorem{definition}{Definition}
\newtheorem{assumption}{Assumption}

% 自定义符号(标准模板+补充)
\newcommand{\ba}{\mathbf{a}}
\newcommand{\bx}{\mathbf{x}}
\newcommand{\bb}{\mathbf{b}}
\newcommand{\bA}{\mathbf{A}}
\newcommand{\m}[1]{\ensuremath{\boldsymbol{#1}}} % 行内公式加粗

\usepackage{algorithm}
\usepackage{algorithmic}

\usepackage{hyperref}
\hypersetup{
	colorlinks=true,
	filecolor=blue,
	citecolor = blue,
	urlcolor=cyan,
}

% 页面布局(完全沿用标准模板,无修改)
\oddsidemargin  0in \evensidemargin 0in \topmargin -0.5in
\headheight 0.2in \headsep 0.2in
\textwidth   6.5in \textheight 9in 
\parskip 1.5ex  \parindent 0ex \footskip 40pt

% 标题加粗设置(标号+文字全加粗,字号匹配原模板)
\titleformat{\section}
{\normalfont\Large\bfseries}{\thesection}{1em}{}
\titleformat{\subsection}
{\normalfont\large\bfseries}{\thesubsection}{1em}{}
\titleformat{\subsubsection}
{\normalfont\normalsize\bfseries}{\thesubsubsection}{1em}{}

% 公式间距调整(移除数学模式内无效的\boldmath和\mathversion命令,解决报错)
\setlength{\abovedisplayskip}{1.0em}  % 公式上方间距
\setlength{\belowdisplayskip}{1.0em}  % 公式下方间距
\setlength{\abovedisplayshortskip}{0.8em}
\setlength{\belowdisplayshortskip}{0.8em}
% 公式加粗调整:仅在文本模式下生效,避免数学模式报错
\AtBeginDocument{
	\mathversion{bold} % 全局数学字体加粗,仅在文档开始时执行一次
}
\everydisplay{} % 移除\boldmath,避免在数学环境内重复执行导致报错

% 列表间距优化(仅调整缩进/行距,无字号)
\setlist[itemize]{leftmargin=2.5em, itemsep=0.6em, align=left} % 新增align=left,优化排版
\setlist[enumerate]{leftmargin=2.5em, itemsep=0.6em, align=left} % 新增align=left,优化排版

\begin{document}
	
	% 标题栏(完全沿用标准模板格式+字号,替换为实验报告标题)
	\framebox[6.4in]{
		\begin{minipage}{6.4in}
			\vspace{1mm}
			\center \makebox[6.2in]{{\bf Optimization algorithms \hfill  11/28/2025}} 
			\vspace{2mm} \\
			\center \makebox[6.2in]{{\Large Lasso回归有效优化算法计算效率对比实验报告}} 
			\vspace{1mm} \\
			\center \makebox[6.2in]{{\it  \hfill Scribe:苗旺\ 佟禹澎\ 孟祥栋}}
			\vspace{1mm}
		\end{minipage}
	} \vspace{2mm} \\
	\mbox{{ \it }}
	
	% 正文内容:严格匹配模板格式,关键小标题加粗
	\section{实验目的}
	Lasso回归实验的核心目标是实现8种针对性有效求解算法,针对不同样本个数$n$与特征维数$p$的组合,对比各算法的\textbf{计算效率}(总耗时、收敛迭代次数)、\textbf{求解精度}(最终目标函数值)及\textbf{稀疏性表现},明确不同应用场景下最优的Lasso求解算法,加深对非光滑优化算法的理解与工程应用能力。
	
	\section{实验原理}
	\subsection{Lasso回归模型定义}
	\begin{definition}
		Lasso(最小绝对收缩和选择算子)是一种带L1正则项的线性回归模型,通过引入L1正则项实现特征选择与系数收缩,其目标函数为非光滑优化问题,数学表达如下:
		\[
		\min_{\beta} J(\beta) = \frac{1}{2n} \|X\beta - y\|_2^2 + \lambda \|\beta\|_1
		\]
		其中$X \in \mathbb{R}^{n \times p}$为特征矩阵,$y \in \mathbb{R}^n$为标签向量,$\beta \in \mathbb{R}^p$为待求回归系数,$\lambda$为正则化参数控制稀疏性强度。
	\end{definition}
	
	该模型中,第一项为最小二乘损失(光滑部分),提供模型拟合能力;第二项为L1正则项(非光滑部分),实现系数稀疏化,是高维小样本场景的核心回归模型之一。
	
	\subsection{有效求解算法筛选}
	因L1正则项的非光滑特性,普通梯度下降、牛顿法等光滑优化算法适配性差,本次实验筛选8种针对性有效算法,核心分类与特性如下:
	\begin{itemize}
		\item \textbf{梯度类算法}:次梯度下降、近端梯度下降、加速梯度下降(适配非光滑目标,软阈值算子处理L1项);
		\item \textbf{逐维度优化算法}:坐标下降(逐维度闭式解,高维场景高效);
		\item \textbf{分布式友好算法}:交替方向乘子法(ADMM,拆分约束问题,稳定性强);
		\item \textbf{大规模数据算法}:随机梯度下降(SGD)、随机方差缩减(SVRG)、Adam(批次计算,降低时间复杂度);
	\end{itemize}
	
	\section{实验配置}
	\subsection{数据生成规则}
	实验采用模拟数据,严格贴合Lasso回归的稀疏应用场景,生成规则如下:
	\begin{enumerate}
		\item \textbf{特征矩阵$X$:}元素服从标准正态分布$X \sim \mathcal{N}(0,1)$,保证数据无偏性;
		\item\textbf{真实系数$\beta$:}仅10\%维度为非零值(随机生成),其余为0,模拟真实场景的稀疏特性;
		\item \textbf{标签向量$y$:}$y = X\beta + 0.1\mathcal{N}(0,1)$,添加小幅高斯噪声,增强实验泛化性;
	\end{enumerate}
	
	\subsection{核心实验配置}
	1.  \textbf{$(n,p)$组合}:覆盖3类典型场景,验证算法在不同数据维度下的表现:
	\begin{itemize}
		\item\textbf{高维小样本:}$(100, 500)$($n < p$,Lasso核心应用场景);
		\item \textbf{等维数据:}$(500, 500)$($n = p$,常规回归场景);
		\item \textbf{低维多样本:}$(1000, 100)$($n > p$,大样本拟合场景);
	\end{itemize}
	
	2.  \textbf{算法参数}:统一收敛标准,保证对比公平性:
	\begin{itemize}
		\item \textbf{正则化参数}$\lambda = 0.1$(固定,控制稀疏性强度一致);
		\item \textbf{最大迭代次数}$10^4$,收敛阈值$10^{-6}$(判断算法终止条件);
		\item \textbf{学习率:}SGD/Adam取0.001,其余算法取0.01(适配各算法特性);
	\end{itemize}
	
	3.  \textbf{评估指标}:从4个维度全面评估算法性能:
	\begin{itemize}
		\item \textbf{总耗时(秒):}算法从初始化到收敛的总计算时间;
		\item \textbf{迭代次数:}算法收敛所需迭代步数(反映收敛速度);
		\item \textbf{最终目标函数值:}算法收敛后的目标函数结果(反映求解精度);
		\item \textbf{系数稀疏性:}回归系数中零值占比(反映L1正则项的稀疏化效果);
	\end{itemize}
	
	\section{实验结果与分析}
	\subsection{实验数据汇总}
	8种算法在3类$(n,p)$组合下的核心指标如下表所示(完整数据见附录),为后续图片分析提供数值支撑:
	
	\begin{table}[htbp]
		\centering
		\caption{各算法核心性能指标汇总}
		\setlength{\tabcolsep}{10pt}
		\begin{tabular}{c c c c c c c}
			\toprule[1.5pt]
			$n$ & $p$ & 算法 & 总耗时(s) & 迭代次数 & 最终目标函数值 & 稀疏性(零系数占比) \\
			\midrule
			\multirow{8}{*}{100} & \multirow{8}{*}{500} & Subgradient Descent & 0.099 & 716 & 3.310 & 0.000 \\
			& & Proximal Gradient & 0.993 & 7027 & 3.119 & 0.808 \\
			& & Accelerated Gradient & 0.053 & 390 & 3.116 & 0.814 \\
			& & Coordinate Descent & 2.894 & 130 & 3.116 & 0.816 \\
			& & ADMM & 0.124 & 228 & 3.349 & 0.000 \\
			& & SGD & 0.187 & 1831 & 3.921 & 0.000 \\
			& & SVRG & 1.604 & 353 & 3.128 & 0.000 \\
			& & Adam & 0.206 & 2058 & 3.362 & 0.000 \\
			\midrule
			\multirow{8}{*}{500} & \multirow{8}{*}{500} & Subgradient Descent & 0.254 & 1025 & 4.470 & 0.000 \\
			& & Proximal Gradient & 0.299 & 1255 & 4.449 & 0.902 \\
			& & Accelerated Gradient & 0.045 & 187 & 4.450 & 0.902 \\
			& & Coordinate Descent & 0.277 & 7 & 4.449 & 0.902 \\
			& & ADMM & 0.453 & 79 & 5.005 & 0.000 \\
			& & SGD & 1.825 & 10000 & 4.599 & 0.000 \\
			& & SVRG & 0.717 & 131 & 4.451 & 0.000 \\
			& & Adam & 0.786 & 2789 & 4.561 & 0.000 \\
			\midrule
			\multirow{8}{*}{1000} & \multirow{8}{*}{100} & Subgradient Descent & 0.140 & 730 & 0.766 & 0.000 \\
			& & Proximal Gradient & 0.115 & 593 & 0.762 & 0.910 \\
			& & Accelerated Gradient & 0.020 & 119 & 0.762 & 0.910 \\
			& & Coordinate Descent & 0.026 & 4 & 0.762 & 0.910 \\
			& & ADMM & 0.055 & 106 & 0.833 & 0.000 \\
			& & SGD & 0.619 & 2944 & 0.791 & 0.000 \\
			& & SVRG & 0.691 & 147 & 0.762 & 0.000 \\
			& & Adam & 0.328 & 1893 & 0.808 & 0.000 \\
			\bottomrule[1.5pt]
		\end{tabular}
	\end{table}
	
	\subsection{计算效率分析(总耗时)}
	各算法在不同$(n,p)$组合下的总耗时直观对比见图1、图2、图3,可清晰观察算法在不同场景下的时间性能差异:
	
	% 图片1:高维小样本(100,500)耗时对比
	\begin{figure}[htbp]
		\centering
		\includegraphics[width=0.8\linewidth]{C:/Users/DELL/Desktop/图片1:高维小样本(100,500)耗时对比} % 替换为你的图片实际文件名
		\caption{Lasso算法耗时对比($n=100, p=500$)}
		\label{fig:time100_500}
	\end{figure}
	
	\textbf{图1显示,}在高维小样本场景下,加速梯度下降耗时最优(仅0.053s),坐标下降耗时最长(2.894s),这是因为高维场景下逐维度优化的累积计算量显著增加。
	
	% 图片2:等维数据(500,500)耗时对比
	\begin{figure}[htbp]
		\centering
		\includegraphics[width=0.8\linewidth]{C:/Users/DELL/Desktop/图片2:等维数据(500,500)耗时对比} % 替换为你的图片实际文件名
		\caption{Lasso算法耗时对比($n=500, p=500$)}
		\label{fig:time500_500}
	\end{figure}
	
	\textbf{图2表明,}等维数据场景下,坐标下降与加速梯度下降耗时均处于较低水平(0.277s vs 0.045s),而SGD因迭代次数接近上限,耗时达到1.825s,成为该场景下效率最低的算法。
	
	% 图片3:低维多样本(1000,100)耗时对比
	\begin{figure}[htbp]
		\centering
		\includegraphics[width=0.8\linewidth]{C:/Users/DELL/Desktop/图片3:低维多样本(1000,100)耗时对比} % 替换为你的图片实际文件名
		\caption{Lasso算法耗时对比($n=1000, p=100$)}
		\label{fig:time1000_100}
	\end{figure}
	
\textbf{图3清晰呈现,}低维多样本场景下,加速梯度下降(0.020s)与坐标下降(0.026s)耗时几乎持平,均远低于其他算法,体现了两类算法在小维度场景下的极致效率。
	
	\textbf{耗时分析核心结论}:
	
	1.  加速梯度下降在所有场景下耗时均最优,是Lasso回归的通用高效算法;
	
	2.  坐标下降在低维/等维场景耗时极短,但在高维小样本场景性能衰减明显;
	
	3.  SGD类算法随样本量增加,耗时增长显著,效率劣势凸显。
	
	\subsection{收敛速度分析(迭代次数)}
\textbf{	收敛速度直接反映算法的迭代效率,结合数值结果与收敛曲线(图4),可得到核心规律:}
	
	% 图片4:收敛曲线对比(以100,500为例,核心场景)
	\begin{figure}[htbp]
		\centering
		\includegraphics[width=0.8\linewidth]{C:/Users/DELL/Desktop/图片4:收敛曲线对比(以100,500为例,核心场景)} % 替换为你的图片实际文件名
		\caption{Lasso有效算法收敛曲线对比($n=100, p=500$)}
		\label{fig:convergence}
	\end{figure}
	
	图4显示,坐标下降的收敛速度最快(仅130次迭代收敛),加速梯度下降次之(390次),而SGD与Adam迭代次数均超1800次,收敛速度最慢。核心结论如下:
	\begin{itemize}
		\item 坐标下降迭代次数最少,因逐维度闭式解无需复杂梯度迭代;
		\item 加速梯度下降借助Nesterov加速,迭代次数适中且单次耗时低,综合效率最优;
		\item 随机梯度类算法因梯度方差大,需更多迭代次数才能收敛。
	\end{itemize}
	
	\subsection{求解精度分析(最终目标函数值)}
	\textbf{最终目标函数值反映算法的求解最优性,结合数值结果与图表趋势,核心结论如下:}
	
	1.  坐标下降、近端梯度下降、加速梯度下降的目标函数值最为接近(误差$<10^{-3}$),求解精度最优;
	
	2.  ADMM在所有场景下目标函数值最大,求解精度最差,因算法通过拆分问题牺牲部分精度换取稳定性;
	
	3.  SGD目标函数值普遍偏高,因随机梯度的随机性导致无法收敛到全局最优解附近。
	
	\subsection{稀疏性分析}
	\textbf{Lasso回归的核心特性是系数稀疏化,各算法的稀疏性表现见图5、图6、图7,直观呈现不同算法的稀疏化能力:}
	
	% 图片5:高维小样本(100,500)稀疏性对比
	\begin{figure}[htbp]
		\centering
		\includegraphics[width=0.8\linewidth]{C:/Users/DELL/Desktop/图片5:高维小样本(100,500)稀疏性对比} % 替换为你的图片实际文件名
		\caption{Lasso算法稀疏性对比($n=100, p=500$)}
		\label{fig:sparsity100_500}
	\end{figure}
	
	% 图片6:等维数据(500,500)稀疏性对比
	\begin{figure}[htbp]
		\centering
		\includegraphics[width=0.8\linewidth]{C:/Users/DELL/Desktop/图片6:等维数据(500,500)稀疏性对比} % 替换为你的图片实际文件名
		\caption{Lasso算法稀疏性对比($n=500, p=500$)}
		\label{fig:sparsity500_500}
	\end{figure}
	
	% 图片7:低维多样本(1000,100)稀疏性对比
	\begin{figure}[htbp]
		\centering
		\includegraphics[width=0.8\linewidth]{C:/Users/DELL/Desktop/图片7:低维多样本(1000,100)稀疏性对比} % 替换为你的图片实际文件名
		\caption{Lasso算法稀疏性对比($n=1000, p=100$)}
		\label{fig:sparsity1000_100}
	\end{figure}
	
\textbf{	图5-图7一致表明:}

	1.  近端梯度下降、加速梯度下降、坐标下降的稀疏性最优(零系数占比0.808-0.910),完美契合L1正则项的稀疏化要求;
	
	2.  次梯度下降、ADMM、SGD、SVRG、Adam的稀疏性为0,说明这类算法虽能求解Lasso目标函数,但无法有效触发L1正则的阈值效应,稀疏化效果失效;
	
	3.  随着特征维数$p$降低,算法稀疏性略有提升,说明低维数据更易通过L1正则实现系数稀疏化。
	
	\section{实验结论}
	\subsection{算法场景化最优选择}
	\textbf{结合7张图片的直观对比与数值分析,不同场景下的最优算法明确如下:}
	
	1.  \textbf{高维小样本($n < p$)}:最优算法为加速梯度下降(耗时最少、精度高、稀疏性好),次优为近端梯度下降;
	
	2.  \textbf{等维数据($n = p$)}:最优算法为坐标下降(迭代次数最少、精度最优),次优为加速梯度下降;
	
	3.  \textbf{低维多样本($n > p$)}:最优算法为坐标下降(耗时最短、迭代最少),次优为加速梯度下降。
	
	\subsection{核心关键发现}
	1. 加速梯度下降是\textbf{综合性能最优}的通用算法,在所有场景下表现均衡,无需针对场景调整,适合工程落地;
	
	2.  坐标下降是\textbf{低维/等维场景专属最优算法},但高维场景性能衰减明显,适用范围具有局限性;
	
	3.  SGD类算法(SGD、SVRG、Adam)稀疏化效果失效,仅适合对稀疏性无要求的大规模数据场景;
	
	4.  ADMM求解精度与稀疏性均不佳,在Lasso回归中无明显优势,更适合分布式大规模任务。
	
	\subsection{实验局限与改进方向}
	\begin{itemize}
		\item 实验数据为模拟数据,未来可使用真实数据集(如基因数据、房价数据)验证算法鲁棒性;
		\item 正则化参数$\lambda$固定为0.1,可进一步研究$\lambda$取值对算法性能的影响;
		\item 算法参数采用默认值,可通过网格搜索优化参数,进一步提升算法性能;
	\end{itemize}
	
	\section{附录:完整实验数据}
	\begin{table}[htbp]
		\centering
		\caption{8种算法完整实验结果}
		\setlength{\tabcolsep}{8pt}
		\resizebox{\linewidth}{!}{
			\begin{tabular}{c c c c c c c}
				\toprule[1.5pt]
				$n$ & $p$ & 算法 & 总耗时(s) & 迭代次数 & 最终目标函数值 & 稀疏性(零系数占比) \\
				\midrule
				100 & 500 & Subgradient Descent & 0.09899759292602539 & 716 & 3.310302870355787 & 0 \\
				100 & 500 & Proximal Gradient & 0.9933252334594727 & 7027 & 3.1191018929054573 & 0.808 \\
				100 & 500 & Accelerated Gradient & 0.053171634674072266 & 390 & 3.1164456360583053 & 0.814 \\
				100 & 500 & Coordinate Descent & 2.8941540718078613 & 130 & 3.116276587730093 & 0.816 \\
				100 & 500 & ADMM & 0.12413811683654785 & 228 & 3.3490657979527647 & 0 \\
				100 & 500 & SGD & 0.18655967712402344 & 1831 & 3.9208600156267264 & 0 \\
				100 & 500 & SVRG & 1.6042168140411377 & 353 & 3.1279514113991276 & 0 \\
				100 & 500 & Adam & 0.20615291595458984 & 2058 & 3.3620138142335283 & 0 \\
				\midrule
				500 & 500 & Subgradient Descent & 0.25441598892211914 & 1025 & 4.470488479756645 & 0 \\
				500 & 500 & Proximal Gradient & 0.2988595962524414 & 1255 & 4.44934448427763 & 0.902 \\
				500 & 500 & Accelerated Gradient & 0.04520297050476074 & 187 & 4.449606659012328 & 0.902 \\
				500 & 500 & Coordinate Descent & 0.27715301513671875 & 7 & 4.449242976545318 & 0.902 \\
				500 & 500 & ADMM & 0.45335888862609863 & 79 & 5.00457320306401 & 0 \\
				500 & 500 & SGD & 1.8251399993896484 & 10000 & 4.598728377335302 & 0 \\
				500 & 500 & SVRG & 0.7170381546020508 & 131 & 4.451327908260974 & 0 \\
				500 & 500 & Adam & 0.7864501476287842 & 2789 & 4.561227159393853 & 0 \\
				\midrule
				1000 & 100 & Subgradient Descent & 0.13953495025634766 & 730 & 0.7662144014317286 & 0 \\
				1000 & 100 & Proximal Gradient & 0.11513280868530273 & 593 & 0.7616944637064024 & 0.91 \\
				1000 & 100 & Accelerated Gradient & 0.020400047302246094 & 119 & 0.7619638094715288 & 0.91 \\
				1000 & 100 & Coordinate Descent & 0.02578449249267578 & 4 & 0.7616383614436414 & 0.91 \\
				1000 & 100 & ADMM & 0.05460190773010254 & 106 & 0.8333249213934252 & 0 \\
				1000 & 100 & SGD & 0.6186990737915039 & 2944 & 0.7911457677822865 & 0 \\
				1000 & 100 & SVRG & 0.6907351016998291 & 147 & 0.762094667718921 & 0 \\
				1000 & 100 & Adam & 0.3281066417694092 & 1893 & 0.8078179255942031 & 0 \\
				\bottomrule[1.5pt]
			\end{tabular}
		}
	\end{table}
	
\end{document}